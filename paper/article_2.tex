%%%%%%%%%%%%%%%%%%%%%%%%%%%%%%%%%%%%%%%%%
% Journal Article
% LaTeX Template
% Version 1.4 (15/5/16)
%
% This template has been downloaded from:
% http://www.LaTeXTemplates.com
%
% Original author:
% Frits Wenneker (http://www.howtotex.com) with extensive modifications by
% Vel (vel@LaTeXTemplates.com)
%
% License:
% CC BY-NC-SA 3.0 (http://creativecommons.org/licenses/by-nc-sa/3.0/)
%
%%%%%%%%%%%%%%%%%%%%%%%%%%%%%%%%%%%%%%%%%

%----------------------------------------------------------------------------------------
%	PACKAGES AND OTHER DOCUMENT CONFIGURATIONS
%----------------------------------------------------------------------------------------

\documentclass[twoside,twocolumn]{article}

\usepackage{blindtext} % Package to generate dummy text throughout this template 

\usepackage[sc]{mathpazo} % Use the Palatino font
\usepackage[T1]{fontenc} % Use 8-bit encoding that has 256 glyphs
\linespread{1.05} % Line spacing - Palatino needs more space between lines
\usepackage{microtype} % Slightly tweak font spacing for aesthetics

\usepackage[english]{babel} % Language hyphenation and typographical rules

\usepackage[hmarginratio=1:1,top=32mm,columnsep=20pt]{geometry} % Document margins
\usepackage[hang, small,labelfont=bf,up,textfont=it,up]{caption} % Custom captions under/above floats in tables or figures
\usepackage{booktabs} % Horizontal rules in tables

\usepackage{lettrine} % The lettrine is the first enlarged letter at the beginning of the text

\usepackage{enumitem} % Customized lists
\setlist[itemize]{noitemsep} % Make itemize lists more compact

\usepackage{abstract} % Allows abstract customization
\renewcommand{\abstractnamefont}{\normalfont\bfseries} % Set the "Abstract" text to bold
\renewcommand{\abstracttextfont}{\normalfont\small\itshape} % Set the abstract itself to small italic text

\usepackage{titlesec} % Allows customization of titles
\renewcommand\thesection{\Roman{section}} % Roman numerals for the sections
\renewcommand\thesubsection{\roman{subsection}} % roman numerals for subsections
\titleformat{\section}[block]{\large\scshape\centering}{\thesection.}{1em}{} % Change the look of the section titles
\titleformat{\subsection}[block]{\large}{\thesubsection.}{1em}{} % Change the look of the section titles

\usepackage{fancyhdr} % Headers and footers
\pagestyle{fancy} % All pages have headers and footers
\fancyhead{} % Blank out the default header
\fancyfoot{} % Blank out the default footer
\fancyhead[C]{Running title $\bullet$ May 2016 $\bullet$ Vol. XXI, No. 1} % Custom header text
\fancyfoot[RO,LE]{\thepage} % Custom footer text

\usepackage{titling} % Customizing the title section

\usepackage{hyperref} % For hyperlinks in the PDF

%----------------------------------------------------------------------------------------
%	TITLE SECTION
%----------------------------------------------------------------------------------------

\setlength{\droptitle}{-4\baselineskip} % Move the title up

\pretitle{\begin{center}\Huge\bfseries} % Article title formatting
\posttitle{\end{center}} % Article title closing formatting
\title{Article Title} % Article title
\author{%
\textsc{John Smith}\thanks{A thank you or further information} \\[1ex] % Your name
\normalsize University of California \\ % Your institution
\normalsize \href{mailto:john@smith.com}{john@smith.com} % Your email address
%\and % Uncomment if 2 authors are required, duplicate these 4 lines if more
%\textsc{Jane Smith}\thanks{Corresponding author} \\[1ex] % Second author's name
%\normalsize University of Utah \\ % Second author's institution
%\normalsize \href{mailto:jane@smith.com}{jane@smith.com} % Second author's email address
}
\date{\today} % Leave empty to omit a date
\renewcommand{\maketitlehookd}{%
\begin{abstract}
\noindent The recent tensions on the measured value of the Hubble constant between CMB and astrophisical observations, has triggered the need of new methods for its determination. In view of this, an effort has been done by H0LiCoW to use the gravitational lensing of quasars as a probe for H0. This type of measurement requires a long term monitoring of lensed quasars (of the order of years). Since big telescopes have to deal with many observational requests, it is difficult to have a constant monitoring over the years, therefore this task can be achieved more easily by small/medium size telescopes. However, the number of lensed quasars with multiple images that can be resolved by these telescopes drops drastically. Here we present a method to deal with \textbf{non resolved} lensed quasars. This method has also the advantage of being less dependent on the microlensing effect of the lens galaxy.
\end{abstract}
}

%----------------------------------------------------------------------------------------

\begin{document}

% Print the title
\maketitle

%----------------------------------------------------------------------------------------
%	ARTICLE CONTENTS
%----------------------------------------------------------------------------------------

\section{Introduction}
In the last years, the precision of the Planck [cita] experiment, whose task was to analyse the Cosmic Microwave Background (CMB) anisotropies, has allowed to fully test our standard cosmological model $\Lambda$CDM which assumes the existence of Dark Energy ($\Lambda$) and Cold Dark Matter (CDM). In particular, in addition to the minimal 6 parameters describing $\Lambda$CDM, the CMB anisotropies allow to indirectly constrain other parameters, such as the current expansion rate of the Universe, $H_0$. 

%------------------------------------------------

\section{Methods}

Maecenas sed ultricies felis. Sed imperdiet dictum arcu a egestas. 
\begin{itemize}
\item Donec dolor arcu, rutrum id molestie in, viverra sed diam
\item Curabitur feugiat
\item turpis sed auctor facilisis
\item arcu eros accumsan lorem, at posuere mi diam sit amet tortor
\item Fusce fermentum, mi sit amet euismod rutrum
\item sem lorem molestie diam, iaculis aliquet sapien tortor non nisi
\item Pellentesque bibendum pretium aliquet
\end{itemize}
\blindtext % Dummy text

Text requiring further explanation\footnote{Example footnote}.

%------------------------------------------------

\section{Results}

\begin{table}
\caption{Example table}
\centering
\begin{tabular}{llr}
\toprule
\multicolumn{2}{c}{Name} \\
\cmidrule(r){1-2}
First name & Last Name & Grade \\
\midrule
John & Doe & $7.5$ \\
Richard & Miles & $2$ \\
\bottomrule
\end{tabular}
\end{table}

\blindtext % Dummy text

\begin{equation}
\label{eq:emc}
e = mc^2
\end{equation}

\blindtext % Dummy text

%------------------------------------------------

\section{Discussion}

\subsection{Subsection One}

A statement requiring citation \cite{Figueredo:2009dg}.
\blindtext % Dummy text

\subsection{Subsection Two}

\blindtext % Dummy text

%----------------------------------------------------------------------------------------
%	REFERENCE LIST
%----------------------------------------------------------------------------------------

\begin{thebibliography}{99} % Bibliography - this is intentionally simple in this template

\bibitem[Figueredo and Wolf, 2009]{Figueredo:2009dg}
Figueredo, A.~J. and Wolf, P. S.~A. (2009).
\newblock Assortative pairing and life history strategy - a cross-cultural
  study.
\newblock {\em Human Nature}, 20:317--330.
 
\end{thebibliography}

%----------------------------------------------------------------------------------------

\end{document}
